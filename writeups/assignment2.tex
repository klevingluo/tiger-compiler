\documentclass{notes}

\title{Compilers Assignment 2}
\author{Kevin Luo \& Brian Roels}
\usepackage{amsmath} % math
\usepackage{graphicx} % graphics
\usepackage{algorithm} % pseodocode
\usepackage{amsfonts} % fonts
\usepackage{amsthm} % theorem formatting
\usepackage[noend]{algpseudocode} % theorem formatting

\begin{document}
  \pagenumbering{gobble}
  \maketitle
  \newpage
  \tableofcontents
  \newpage
  \pagenumbering{arabic}

\section{notes}
Since LR parsers use constant stack when recursion in rules is left,
we do that whenever possible, like the arguments to a function, a
sequence of expressions, the fields of a record, etc

some shift-reduce conflicts we resovled and how:
multiplication and division bind tighter than addition and
subtraction. All all left-associative, although this only matters for
subtraction and division.

Relative operators, <, > and the like ,bind tighter than boolean
operators, but not as tightly as arith ops

assignment binds loosest than all of the above.

DO binds tighter than operators, since a while loop's body does not
produce a value, although it seems that calling an operator an this
value will cause problems during typechecking anyway

operators bind more tightly than then's , since we assume that we want
to return the sum if the atatement is true.  This one is not so cut
and dry ,b ut we think this is the right decision.

we resolve the if then elses to the innermost statement by making then
of higher precedence then else

when creating an array, we make operators of higher precedence on the
chain, so that the calculated value is put into the array.

we prioritize id over lbrack, meaning we choose not to reduce to an
lvalue

we use the precedence rules for each operator when reducing
expressions with operators.

we have separate cases for the empty record type and value, which is
different from the empt record type value

\end{document}
